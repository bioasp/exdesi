
\section{Discussion}

We presented a new approach to experiment planning with multiple perturbations
in the context of interaction graph models.
We demonstrated \emph{in silico} that our planning approach proposes experiments
that are suitable to restore a gold standard model from a distorted model.
Further, we showed \emph{in vivo} that our approach proposes experiments that
allowed us to systematically reduce our space of possible models for the
Erythropoietin signal transduction in \texttt{HEK293} cells and
to exclude certain interactions.

In an empirical study, we showed that while the approach is sensitive to the number of possible perturbations,
it works robustly on problems of realistic sizes.
For all the tested candidate sets at least one discriminating experiment could be proposed.
\texttt{ExDesi} is easily scalable in the number of possible readouts which, as expected,
also leads to much better results.
In contrast, the approach has only limited scalability in the number of possible perturbations
since increasing those leads to an exponential growth of the search space.
We are confident that future research and the progress in the used solver technology
will allow us to scale this approach to problems with larger search spaces.

Generally, as holds true for many model discrimination methods,
models which are underconstrained (e.g., because there are too many candidate edges in the initial graph)
can usually not be falsified because the model explains all observed behaviors.
Thus, it is very important to start with a sparse model
where (most) included interactions have a relatively high confidence.

Only few methods for the selection of optimal sets of multiple perturbations have
been proposed so far.
\texttt{MEED}~\cite{szczurek2009} is an approach that uses ternary logic networks and microarray data.
\texttt{Caspo}~\cite{videla2015} on the other hand uses Boolean logic networks and Boolean data (on and off) within a single state of the biological system.
In contrast, our method is based on interaction graphs and uses data on (signed) changes (ups and downs between two states) as response to perturbations.
\texttt{ExDesi} is to a certain degree related to \texttt{Caspo} as both work on topological similar models (Boolean networks transformed to interaction graphs),
but one cannot easily extract a Boolean state from data expressing signed changes.
For example the sign for an increased value in a variable can be interpreted as a Boolean variable that switches from an OFF state to an ON state,
but depending on the thresholds used to mark the border between ON and OFF it could also refer to a variable that remains in its initial OFF/ON state.

\begin{figure}[]
\begin{center}
  \begin{tikzpicture}[->,>=stealth',scale=1.2]
    \sffamily
    \tikzstyle{label}=[text=black]
    \tikzstyle{wht}=[draw=none, fill=none,text=black,text opacity=1,text=black]
    \tikzstyle{bbox}=[draw=yellow,ellipse,minimum height=3em, text opacity=1,fill=lightyellow,text=black, text badly centered]
    \tikzstyle{box}=[draw=node_blue,rectangle,minimum height=2em,text opacity=1,fill=lightblue,text=black, text badly centered, text width=6em]
    \tikzstyle{decision}=[diamond, draw, fill=red!15, text width=5em, text badly centered, node distance=15ex, inner sep=0pt]


    \node[box] (PED) at (-0.5,3.3)  {\small Prior experiment data};
    \node[box] (PKN) at (4.5,3.5)     {\small Prior network candidates};
    \node[bbox,minimum height=3em, text width=4em] (NI) at (2.5,2.25)  {\small Network inference};
    \node[box] (C) at (2.5,1.0)     {\small New network candidates};
    \node[bbox] (EP) at (2.5,-0.25) {\small Experiment design};
    \node[box]  (PE) at (2.5,-1.6)  {\small $1\dots n$ experiment proposals};
    \node[bbox] (E) at (2.5,-3)   {\small Experimentation};
    \node[box] (ED) at (-0.5,-3)     {\small New experiment data};
%     \node[label] (l) at (0,-1.5)  {};

    \path[every node/.style={anchor=south}]
     (PED) edge[->] (NI)
     (PKN) edge[->] (NI)
     (NI)  edge[->] (C)
     (C)   edge[->] (EP)
     (EP)  edge[->] (PE)
     (PE)  edge[->] (E)
     (E)   edge[->] (ED)
     (ED)  edge[->] (PED);
     \draw[->,rounded corners=1em, thin,]
     (C.east) -- (4.5,1.0) -- (4.5,2.2) -- (PKN.south);
  \end{tikzpicture}

\caption{
Workflow of experiment planning.
}
\label{fig:planning_loop}
\end{center}
\end{figure}

The presented approach to experiment planning integrates nicely in the systems
biology work-flow of experimentation, data analysis, hypothesis generation and
model inference (Figure~\ref{fig:planning_loop}).
One usually starts the process with some prior knowledge and experimental data.
Using network inference methods one obtains one or more candidate models.
These network candidates together with information about already performed
 experiments and possible perturbations and readouts are fed into the
 experiment design process, which proposes new experiments that are suitable to
 discriminate among these candidates.
The resulting experiments are performed and the experimental results are
compared with the predicted model behaviors.
Models whose predicted behaviors are inconsistent with the experimental result
must be discarded and the new experimental data is added to the network
inference process to produce new model candidates.
Ideally, this process can be re-iterated until one is left with a single model
candidate.
In the less optimistic scenario one may be left with a set of indistinguishable
model candidates, then other methods must come into play.

Looking at the broader context of experiment planning we see further
optimization potential,
for example the optimal utilization of resources which so far have not been considered.
When the resources to perform multiple experiments in parallel exist and if the
experiments are time-consuming it will be desirable to conduct multiple experiments
at the same time.



